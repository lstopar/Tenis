\documentclass[a4paper, 11pt]{article}
\usepackage{comment} % enables the use of multi-line comments (\ifx \fi) 
\usepackage{lipsum} %This package just generates Lorem Ipsum filler text. 
\usepackage{fullpage} % changes the margin
\usepackage{amsfonts}
\usepackage{mathtools}

\newcommand*\mean[1]{\bar{#1}}

\begin{document}
%Header-Make sure you update this information!!!!

\section{Hitra razlaga}

Problem nacrtovanja tekem teniske lige resimo z aplikacijo prirejanja na utezenem grafu, ki ga predstavljajo igralci. 

Graf je predstavljen kot $G=(V,E,f)$, kjer je $V$ mnozica tock (igralcev v ligi) v grafu, $E \subseteq V \times V$ mnozica povezav (potencialnih tekem) in $f: E \rightarrow \mathbb{R}$ funkcija, ki doloci utez povezave. Intuitivno vecja kot je vrednost $f(i,j)$ manj je moznosti, da bosta igralca $i$ in $j$ izzrebana.

Algoritem na vhodu vzame mnozico vseh povezav grafa v obliki:
\begin{equation}
	\{(i_1,j_1,f(i_1,j_1)), (i_2,j_2,f(i_2,j_2)), ..., (i_m,j_m,f(i_m,j_m))\}
\end{equation}
ter vrne izzrebane pare igralcev v obliki:
\begin{equation}
	\{(k_1,l_1), (k_2,l_2), ..., (k_n,l_n)\}
\end{equation}

V \ref{sec:calc}. poglavju so podana pravila kako izracunamo utez posamezne povezave, oziroma kdaj povezave ne damo v graf.

%----------------------------------------------------------------------------------------
%	IZRACUN FUNKCIJE UTEZI
%----------------------------------------------------------------------------------------

\section{Izracun utezi za podan par igralcev}
\label{sec:calc}

Izracun cene potencialne tekme (utez povezave) se izracuna kot utezena vsota. Koeficienti $w_1, w_2, ...$ naj se nastavijo skozi uporabniski vmesnik.

Za podana igralca $i$ in $j$ izracunamo ceno medsebojne tekme po sledeci formuli:
\begin{equation}
	f(i,j) = w_1 f_1(i,j) + w_2 f_2(i,j) + w_3 f_3(i,j) + w_4 f_4(i,j) + w_5 f_5(i,j)
\end{equation}

\subsection{Igralca se ne marata}

Ce se igralca $i$ in $j$ ne marata, ne smeta v nobenem primeru igrati med seboj! Zato se povezave $\left(i,j,f(i,j)\right)$ ne damo v graf.

\subsection{Igralca sta ze igrala v istem krogu}

Ce je to druga tekma v krogu in sta igralca $i$ in $j$ ze igrala v privi tekmi tega kroga, potem povezave $\left(i,j,f(i,j)\right)$ ne damo v graf.

\subsection{Kazen za razliko v zmagah $f_1$}

Razlika v zmagah se kaznuje po sledeci formuli:
\begin{equation}
	f_1(i,j) = \left|\frac{1 + z(i)}{2 + z(i) + p(i)} - \frac{1 + z(j)}{2 + z(j) + p(j)}\right|
\end{equation}
kjer $z(i)$ predstavlja stevilo zmag igralca $i$ od zacetka lige in $p(i)$ stevilo porazov igralca $i$ od zacetka lige. Ob pricetku lige velja $z(i) = p(i) = 0 \:\: \forall i$.

\subsection{Kazen za razliko v dobljenih gemih $f_2$}

Razlika v dobljenih gemih se kaznuje po formuli:
\begin{equation}
	f_2(i,j) = \left|\frac{9 + d(i)}{18 + d(i) + z(i)} - \frac{9 + d(j)}{18 + d(j) + z(j)}\right|
\end{equation}
kjer $d(i)$ predstavlja skupno stevilo dobljenih gemov igralca $i$ od zacetka lige in $z(i)$ predstavlja skupno stevilo zgubljenih gemov od zacetka lige.

\subsection{Kazen za medsebojne tekme $f_3$}

\begin{equation}
	f_3(i,j) = n_{ij}
\end{equation}
kjer $n_{ij}$ predstavlja stevilo medsebojnih tekem igralcev $i$ in $j$ od zacetka lige.

\subsection{Kazen za igranje v zadnjem letu $f_4$}

\begin{equation}
	f_4(i,j) = 
		\begin{dcases*}
			\text{1,} & ce sta igralca igrala medsebojno tekmo v zadnjih 12 mesecih \\
			\text{0,} & v nasprotnem primeru
		\end{dcases*}
\end{equation}

\subsection{Popust, ce ima kateri igralec podpovprecno iger $f_5(i,j)$}

\begin{equation}
	f_5(i,j) = 
		\begin{dcases*}
			\text{-1,} & $n(i) < \mean{n}$ ali $n(j) < \mean{n}$ \\
			\text{0,} & v nasprotnem primeru
		\end{dcases*}
\end{equation}
kjer je $n(i)$ stevilo tekem, ki jih je odigral igralec $i$ od zacetka lige in $\mean{n} = \frac{1}{\left|V\right|}\sum_{k \in V}n(k)$ predstavlja povprecno stevilo tekem na igralca.

%----------------------------------------------------------------------------------------
%	VREDNOST UTEZI
%----------------------------------------------------------------------------------------

\section{Prizvete vrednosti koeficientov}
\begin{eqnarray}
	w_1 &=& 0 \\ \nonumber
	w_2 &=& 0 \\ \nonumber
	w_3 &=& 100 \\ \nonumber
	w_4 &=& 10 \\ \nonumber
	w_5 &=& 1
\end{eqnarray}

\end{document}